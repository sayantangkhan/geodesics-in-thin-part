\section{The Weak Convex Core of $\teich(\no_g)$}
\label{sec:weak-convex-core}

\subsection{Issues with Geometric Finiteness and Statistical Convex-Cocompactness}
\label{sec:issu-with-geom}

In order to show that the action of $\mcg(\no_g)$ on $\teich(\no_g)$ is geometrically finite (in the sense of Fuchsian groups), we need to exhibit a \emph{convex core}, i.e.\ a convex subset of $\teich(\no_g)$ on which the action of $\mcg(\no_g)$ is finite covolume.
Similarly, to show that the action of $\mcg(\no_g)$ on $\teich(\no_g)$ is statistically convex-cocompact, we need to exhibit a \emph{statistical convex core}, which is a \emph{statistically convex subset} (see Definition \ref{defn:statistical-convex-subset}) of $\teich(\no_g)$ on which $\mcg(\no_g)$ acts cocompactly.

A candidate for the convex core was suggested by \textcite{gendulphe2017whats}, namely the \emph{one-sided systole superlevel set} $\systole(\no_g)$.

\begin{definition}[One-sided systole superlevel set]
  The one-sided systole superlevel set is the subset of $\teich(\no_g)$ where no one-sided curve is shorter than $\vept$. This set is denoted $\systole(\no_g)$.
\end{definition}

The subset $\systole(\no_g)$ has several properties that suggest it should be the convex core for the $\mcg(\no_g)$ action.
\begin{itemize}
\item[-] The space $\teich(\no_g)$ $\mcg(\no_g)$-equivariantly deformation retracts onto the subset $\systole(\no_g)$ (Proposition 19.2 of \cite{gendulphe2017whats}).
\item[-] The $\mcg(\no_g)$ action on $\systole(\no_g)$ has finite $\nu_N$-covolume, where $\nu_N$ is the non-orientable analog of the Weil-Petersson volume form (Proposition 19.1 of \cite{gendulphe2017whats}).
\end{itemize}

However, the subset $\systole(\no_g)$ fails to be convex, in a very strong sense, as we show in a prior paper.
\begin{theorem}[Theorem 5.2 of \cite{limitsetkhan}]
  For all $\vept > 0$, and all $D > 0$, there exists a Teichmüller geodesic segment whose endpoints lie in $\systole(\no_g)$ such that some point in the interior of the geodesic is more than distance $D$ from $\systole(\no_g)$.
\end{theorem}

With this obstruction, we see that $\systole(\no_g)$ will not work as a convex core for a geometrically finite action.
There are two directions one could go in with this obstruction in mind, which we phrase as open questions.

If we wish to show that $\mcg(\no_g)$ acts geometrically finitely, this is the question we need to answer.
\begin{question}
  \label{ques:geom-finite-core}
  Does there exist some other subset of $\teich(\no_g)$ that is finite $\nu_N$-covolume, convex, and an $\mcg(\no_g)$-equivariant deformation retract of $\teich(\no_g)$?
\end{question}
% For instance, one could define a subset $\wt{\systole}(\no_g)$, which is the subset of $\teich(\no_g)$ where the \emph{extremal length} of no one-sided curve is shorter than $\vept$.
% This subset could pos

Alternatively, if we wish to show that $\mcg(\no_g)$ acts statistically convex-cocompactly on $\teich(\no_g)$, where $\thick(\no_g)$ acts as the statistical convex core, this is the question we need to answer.

\begin{question}
  \label{ques:full-stat-core}
  Is $\thick(\no_g)$ statistically convex?
\end{question}

We suspect the answer to Question \ref{ques:full-stat-core} is yes, despite our methods not working.
Our methods for proving statistical convexity rely on proving recurrence of random walks on $\teich(\no_g)$, and random walks on all of $\teich(\no_g)$ have poor recurrence properties when they enter regions where one-sided curves are short.
When the random walks enter these one-sided thin regions, they behave like symmetric random walks on $\mathbb{Z}$, which we know do not have very strong recurrence properties.
We explain this in more detail in Section \ref{sec:why-approach-fails}, when we set up the machinery of Foster-Lyapunov-Margulis functions.

\subsection{A Weaker Notion of Convexity}
\label{sec:weak-noti-conv}

Rather than directly answering questions \ref{ques:geom-finite-core} or \ref{ques:full-stat-core}, we define an even weaker notion of convexity as an intermediate goal.
In the next subsection, we will show that $\systole(\no_g)$ satisfies this weaker notion of convexity.
In this section, we define the notion, and explain why this weaker notion of convexity is still sufficient for the purposes of Patterson-Sullivan theory.

\begin{definition}[Weak convexity]
  A subset $S$ of a geodesic metric space $X$ is said to be $\vepd$-weak convex (for $\vepd > 0$) if there exists a constant $t > 0$ such that for any pair of points $x$ and $y$ in $S$, any geodesic path $\gamma$ joining $x$ and $y$ longer than $t$ can be homotoped to a path $\gamma^{\prime}$ joining $x$ and $y$ such that $\gamma^{\prime}$ lies entirely within $S$, and the lengths of $\gamma$ and $\gamma^{\prime}$ satisfy the following inequality.
  \begin{align*}
    \ell(\gamma^{\prime}) \leq (1 + \vepd) \ell(\gamma)
  \end{align*}
\end{definition}

\begin{remark}
  Strictly speaking, we should call a subset $(\vepd, t)$-weak convex, as the constant $t$ is part of the data that makes a set weak convex.
  However, the constant $t$ will not matter for us, so we suppress it in all mentions of weak convexity.
\end{remark}

An $\vepd$-weak convex subset is a subset which, while not entirely undistorted, has bounded distortion with respect to the ambient metric space at large enough scale.
% When $\vepd = 0$, an $\vepd$-weak convex is an actual convex subset, at least for uniquely geodesic metric spaces.
Weak convexity also interacts well with results from Patterson-Sullivan theory.
Suppose we have a discrete group $G$ acting properly discontinuously on a metric space $X$, and let $X_{\vepd}$ be an $\vepd$-weak convex subset of $X$ upon which $G$ also acts.
If the critical exponent for the $G$ action on $X$ is $\delta$, and the corresponding exponent for $X_{\vepd}$ is $\delta_{\vepd}$, we immediately get the following estimate for $\delta$.
\begin{align*}
  \delta \leq \delta_{\vepd} (1 + \vepd)
\end{align*}

% We can also define a \emph{slowed down geodesic flow} for a subset of the tangent directions on $X_{\vepd}$: namely the tangent directions along which the geodesic enters $X_{\vepd}$ infinitely often in either direction.
% For such bi-infinite geodesics, we can homotope them to lie entirely within $X_{\vepd}$, by only increasing their length by $\vepd$ fraction.
For tangent directions along which the Teichmüller geodesic stays in $\systole(\no_g)$ for arbitrarily large times, we can consider the Teichmüller geodesic flow, and reparameterize the flow speed such that the following equation holds.
\begin{align*}
  d_{\vept}(v, g_{\tau} v) = \tau
\end{align*}
If we can establish mixing results for this reparameterized geodesic flow, we can use the techniques of \textcite{roblin2003ergodicite} to count lattice points in $X_{\vepd}$, the counting results translate into estimates for lattice points in $X$.
\begin{align*}
  \#\left( \text{Lattice points in $X$ within distance $R$} \right) \leq \#\left( \text{Lattice points in $X_{\vepd}$ within distance $R(1 + \vepd)$} \right)
\end{align*}

We can get even better estimates if rather than having a single $\vepd$-weak convex subset, we have an family of subsets, such that the $\vepd$ goes to $0$, and the union of the weak convex subsets is the entire space.

\begin{definition}[Exhaustion by weak convex subsets]
  A metric space $X$ is said to be exhausted by weak convex subsets if there exists a nested family of subsets $\left\{ X_i \right\}$, such that $X_i$ is $\varepsilon_i$-weak convex, where $\varepsilon_i$ goes to $0$, and $\bigcup_{i=1}^{\infty} X_i = X$.
\end{definition}

When there is an exhaustion by weak convex subsets, one can get arbitrarily good bounds for the critical exponent for the $G$ action on $X$.
% , as well as count lattice points on $X$ with only subexponential error terms.

\subsection{Weak Convexity for $\systole(\no_g)$}
\label{sec:weak-conv-syst}

In this section, we will show that $\systole(\no_g)$ is an $\vepd$-weak convex subset of $\teich(\no_g)$, and that $\teich(\no_g)$ can be exhausted by the subsets $\systole(\no_g)$ as $\vept$ goes to $0$.

\begin{theorem}
  \label{thm:weak-convexity}
  For any $\vepd > 0$, there exists a $\vept > 0$ such that $\systole(\no_g)$ is a $\vepd$-weak convex subset of $\teich(\no_g)$.
\end{theorem}

The key ingredient in the proof of Theorem \ref{thm:weak-convexity} is a version of Minsky's product region theorem \cite[Theorem 6.1]{1077244446} for non-orientable surfaces, which we prove in Section \ref{sec:minskys-prod-regi}.

Let $\gamma = \left\{ \gamma_1, \ldots, \gamma_j, \ldots, \gamma_k \right\}$ be a multicurve on a non-orientable surface $\no_g$, where for $i \leq j$, $\gamma_i$ is a two-sided curve, and for $i > j$, $\gamma_i$ is a one-sided curve.
Let $X_\gamma$ denote the metric space obtained as the $\sup$-product $\teich(\no_g \setminus \gamma) \times \mathbb{H}_1 \times \cdots \times \mathbb{H}_j \times (\mathbb{R}_{>0})_{j+1} \times \cdots \times (\mathbb{R}_{>0})_k$, where the $\mathbb{H}_i$ are copies of the upper half plane with the hyperbolic metric, and $\mathbb{R}_{>0}$ is the set of positive real numbers, where the distance between $x$ and $y$ is $\left| \log \left( \frac{x}{y} \right) \right|$.
For any pants decomposition that contains $\gamma$, we consider the Fenchel-Nielsen coordinate systems associated to the pants decomposition.
We have a map $\Pi$ from $\teich(\no_g)$ to $X_\gamma$, which is called the \emph{product region projection map}.
\begin{definition}[Product region projection map]
  The product region projection map $\Pi: \teich(\no_g) \to X_\gamma$ is defined in the following manner.
  \begin{itemize}
  \item The $\teich(\no_g \setminus \gamma)$-coordinate is obtained by setting the lengths to $0$ of all the curves in $\gamma$ to get a punctured hyperbolic surface.
  \item The $\mathbb{H}_i$-coordinate is $\left( t, \frac{1}{\ell} \right)$, where $t$ is the \emph{twist} (i.e. the twist coordinate in the Fenchel-Nielsen coordinate system) of the two-sided curve $\gamma_i$, and $\ell$ is the hyperbolic length.
  \item The $(\mathbb{R}_{>0})_i$ coordinate is $\frac{1}{\ell}$, where $\ell$ is the hyperbolic length of the one-sided curve $\gamma_i$.
  \end{itemize}
\end{definition}

We define a metric on the product $X_\gamma$ as the supremum of the metrics on each of the components, where the metric on $\teich(\no_g \setminus \gamma)$ is the Teichmüller metric, the metric on the $\mathbb{H}_i$ components is the hyperbolic metric, and the metric on the $(\mathbb{R}_{>0})_i$ is given by $d(x,y) = \left| \log\left( \frac{x}{y}  \right) \right| $, i.e.\ the restriction of the hyperbolic metric in $\mathbb{H}$ to a vertical line.

We consider the restriction of $\Pi$ to the thin region of Teichmüller space, denoted $\teich_{\gamma \leq \vept}(\no_g)$, which is the region where all curves in $\gamma$ have hyperbolic length at most $\vept$.

\begingroup
\def\thetheorem{\ref{thm:prno}}
\begin{theorem}[Product region theorem for non-orientable surfaces]
  For any $c >0$, there exists  $\vept^{\prime} > 0$, such that for any $\vept < \vept^{\prime}$, the restriction of $\Pi$ to $\teich_{\gamma \leq \vept}(\no_g)$ is an isometry with additive error at most $c$, i.e.\ the following holds for any $x$ and $y$ in $\teich_{\gamma \leq \vept}(\no_g)$.
  \begin{align*}
    \left| d(x, y) - d_{X_{\gamma}}(\Pi(x), \Pi(y)) \right| \leq c
  \end{align*}

\end{theorem}
\addtocounter{theorem}{-1}
\endgroup

We can now prove Theorem \ref{thm:weak-convexity}.

\begin{proof}[Proof of Theorem \ref{thm:weak-convexity}]
  We begin by picking a small constant $\vept^{\prime} > 0$ and $\delta > 0$.
  We will fix the values of these constants at the end of the proof.
  Let $[x,y]$ be a geodesic segment that starts and ends in $\systolearg{\vept^{\prime}}(\no_g)$.
  Let $\{p_i\}$ be points on $[x,y]$, such that $x = p_0$, $d(p_i, p_{i+1}) = \delta$, and $d(p_n, y) \leq \delta$, where $p_n$ is the last of the $p_i$'s.

  The first step of our proof is modifying the path $[x,y]$ and estimating the length of the modified path.
  We do so by constructing new points $p_i^{\prime}$, where $p_i^{\prime}$ is obtained from $p_i$ by increasing the length of any one-sided curve that is shorter than $\vept^{\prime}$ to $\vept^{\prime}$.
  This ensures that the endpoints of the segments $[p_i^{\prime}, p_{i+1}^{\prime}]$ are in $\systolearg{\vept^{\prime}}(\no_g)$.
  Estimating $d(p_i^{\prime}, p_{i+1}^{\prime})$ splits up into two cases.
  \begin{enumerate}[(i)]
  \item When $p_i = p_i^{\prime}$ and $p_{i+1} = p_{i+1}^{\prime}$: In this case $d(p_i^{\prime}, p_{i+1}^{\prime}) = \delta$, by construction.
  \item When at least one of $p_i$ and $p_{i+1}$ are not equal to $p_i^{\prime}$ and $p_{i+1}^{\prime}$: In this case, we can assume without loss of generality that both $p_i \neq p_i^{\prime}$ and $p_{i+1} \neq p_{i+1}^{\prime}$.
    If that is not the case, and say $p_i \neq p_i^{\prime}$ and $p_{i+1} = p_{i+1}^{\prime}$, we replace $p_{i+1}$ with the last point $y$ on $[p_i, p_{i+1}]$ that is outside $\systolearg{\vept^{\prime}}(\no_g)$.
    The interval $[y, p_{i+1}]$ can be treated as in case (i), and we focus on $[p_i, y]$.

    We have that the interior of $[p_i, p_{i+1}]$ and $[p_i^{\prime}, p_{i+1}^{\prime}]$ both lie in the region where some one-sided curve $\gamma$ is shorter than $\vept^{\prime}$.
    We invoke Theorem \ref{thm:prno} to estimate distances in this region: we have a constant $c(\vept^{\prime})$ that depends on $\vept^{\prime}$ such that following holds.
    \begin{align}
      \label{eq:p-estimate}
      \left| d(p_i, p_{i+1}) - \sup\left( d_{\teich(\no_g \setminus \gamma)}(\Pi(p_i), \Pi(p_{i+1})), \left| \log \left( \frac{\ell_{p_i}(\gamma)}{\ell_{p_{i+1}}(\gamma)}  \right) \right| \right) \right| \leq c(\vept^{\prime})
    \end{align}
    Observe that when we replace $p_i$ by $p_i^{\prime}$ and $p_{i+1}$ by $p_{i+1}^{\prime}$, the first argument $\sup$ stays the same, and the second argument becomes $0$.
    \begin{align}
      \label{eq:p-prime-estimate}
      \left| d(p_i^{\prime}, p_{i+1}^{\prime}) - \sup\left( d_{\teich(\no_g \setminus \gamma)}(\Pi(p_i^{\prime}), \Pi(p_{i+1}^{\prime})), 0 \right) \right| \leq c(\vept^{\prime})
    \end{align}
    This leads to the following estimate for $d(p_i^{\prime}, p_{i+1}^{\prime})$.
    \begin{align}
      \label{eq:length-estimate-homotope}
      d(p_i^{\prime}, p_{i+1}^{\prime}) \leq \delta + 2c(\vept^{\prime})
    \end{align}
  \end{enumerate}
  We construct a new path $\lambda$ by joining $p_i^{\prime}$'s, and $p_n$ to $y$.
  If we let $l$ denote the length of $[x,y]$, we get the following estimate for $\ell(\lambda)$ using \eqref{eq:length-estimate-homotope}.
  \begin{align*}
    \ell(\lambda) \leq l \left( 1 + \frac{2c(\vept^{\prime})}{\delta} \right)
  \end{align*}
  We now pick a value of $\delta$ small enough such that along each of the segments $[p_i, p_{i+1}]$, there is at least one one-sided curve that stays short throughout, and then we pick $\vept^{\prime}$ small enough so that $c(\vept^{\prime})$ is small enough to make $\frac{2c(\vept^{\prime})}{\delta} < \vepd$.

  We now need to show that this new path stays within $\systole(\no_g)$ for some $\vept < \vept^{\prime}$.
  We already have that $x$, $y$ and all the $p_{i}^{\prime}$ are in $\systolearg{\vept^{\prime}}(\no_g)$ and thus in $\systole(\no_g)$.
  For the interior of the geodesic segments $[p_i^{\prime}, p_{i+1}^{\prime}]$, since the endpoints are in $\systolearg{\vept^{\prime}}(\no_g)$, and the length of the segments is no more than $\delta(1 + \vepd)$, we have that there exists some $\vept$ such that $[p_i^{\prime}, p_{i+1}^{\prime}]$ lies in $\systole(\no_g)$.

  Finally, we have to deal with geodesic segments $[w,z]$ which start or end in $\systole(\no_g) \setminus \systolearg{\vept^{\prime}}(\no_g)$.
  We do so by increasing the lengths of short one-sided curves on $w$ and $z$ to $\vept^{\prime}$ if there are any curves shorter than $\vept^{\prime}$.
  Let the modified points be $w^{\prime}$ and $z^{\prime}$: we first construct a path joining $w^{\prime}$ and $z^{\prime}$ that stays within $\systole(\no_g)$ using our construction, and then prepend that path with a path joining $w$ with $w^{\prime}$ and append a path joining $z^{\prime}$ to $z$.
  This new path joining $w$ to $z$ stays entirely withing $\systole(\no_g)$, but we now incur a fixed additive error along with our multiplicative error as well.
  However, if the path is long enough, the additive error can be absorbed in the multiplicative error, with a slightly worse constant.
  We do that, and the threshold for the path being long enough is the constant $t$ that appears in our definition of weak convexity.
  This proves the result.
\end{proof}

\begin{remark}
  We emphasize that the key step in the above proof is going from \eqref{eq:p-estimate} to \eqref{eq:p-prime-estimate}, where the $\log\left( \frac{\ell_{p_i}(\gamma)}{\ell_{p_i^{\prime}}(\gamma)}  \right)$ term becomes $0$.
  This is only possible because there cannot be any twisting around a one-sided curve $\gamma$, so the projection map that sends $p_i$ to $p_i^{\prime}$ and $p_{i+1}$ to $p_{i+1}^{\prime}$ is distance reducing.
  If one tried to use the same proof strategy to show that the thick part of $\teich(\os)$, for any orientable or non-orientable surface $\os$ is weak convex in $\teich(\os)$, the step we described would be the point of failure.
  In particular, if there's a twist along $\gamma$, going from \eqref{eq:p-estimate} to \eqref{eq:p-prime-estimate} will not be distance reducing, and will exponentially increase the distance, leading the estimate to fail.
\end{remark}

Now that we have established that $\systole(\no_g)$ is $\vepd$-weak convex, we can justifiably call it the weak convex core of $\teich(\no_g)$.
For the remainder of this paper, we fix $\vepd < \left(  \frac{1}{6g-12} \right)^2$, and a value of $\vept$ such that $\systole(\no_g)$ is $\vepd$-weak convex.

\begin{definition}[Weak convex core of $\teich(\no_g)$]
  We call $\systole(\no_g)$ the weak convex core of $\teich(\no_g)$, and denote it $\core(\teich(\no_g))$.
  % The specific value of $\vept$ will usually not be important (as long as it is smaller than a threshold value), or clear from context.
\end{definition}

We now also provide a partial classification of the \emph{strongly contracting} elements of $\mcg(\no_g)$ for the metric space $\systole(\no_g)$.

\begin{definition}[Strongly contracting element]
  An infinite order element $\gamma$ in $\mcg(\no_g)$ is said to be strongly contracting if there exists a $p \in \systole(\no_g)$ such that the following two conditions hold.
  \begin{enumerate}[(i)]
  \item $\left\{ \gamma^i p \right\}_{i \in \mathbb{Z}}$ quasi-isometrically embeds in $\systole(\no_g)$.
  \item For any ball of radius $R$ disjoint from $\left\{ \gamma^i p \right\}$, its closest point projection onto $\left\{ \gamma^i p \right\}$ has uniformly bounded diameter.
  \end{enumerate}
\end{definition}

\begin{lemma}[Partial classification of strongly contracting elements]
  \label{lem:strongly-contract-class}
  Let $\gamma$ be an infinite order element in $\mcg(\no_g)$.
  \begin{enumerate}[(i)]
  \item If $\gamma$ is pseudo-Anosov, then $\gamma$ is strongly contracting.
  \item If $\gamma$ leaves a two-sided curve invariant, then $\gamma$ is not strongly contracting.
  \end{enumerate}
\end{lemma}

\begin{proof}
  We deal with the two cases separately.
  \begin{description}
  \item[Case (i)] In this case, we pick $p$ to lie along the axis of the pseudo-Anosov $\gamma$.
    Passing to the orientable double cover, we have that the closest point projection of any ball disjoint from the axis has bounded diameter, by \textcite{minsky1996quasi}.
    Since $\teich(\no_g)$ isometrically embeds inside the Teichmüller space of the double cover, we have the claim for $\teich(\no_g)$.
    To show now that the result holds for $\systole(\no_g)$, observe that the induced metric on $\systole(\no_g)$ is minimally distorted from the metric on $\teich(\no_g)$, by Theorem \ref{thm:weak-convexity}.
    One of two things can happen: the axis of $\gamma$ lies in $\systole(\no_g)$, or it lies outside.
    In the first case, we project as usual, and by Theorem \ref{thm:weak-convexity}, the size of the projection increases by a bounded multiplicative factor.
    In the second case, we first create a new axis, by increasing the lengths of all one-sided curves on the old axis to $\vept$ (while keeping all the other lengths and twists with respect to a Fenchel-Nielsen coordinate system constant): we call this map $\pi$.
    This construction gives us a new axis that lies in $\systole(\no_g)$.
    We then get a projection of a ball by first projecting to the old axis, and then composing that with the $\pi$.
    This composed map still has bounded diameter because the first projection does, and the second projection increases distances by a bounded multiplicative factor.
  \item[Case (ii)] In this case, we need to show there is no choice of $p$ such that the projection onto $\left\{ \gamma^i p \right\}$ has bounded diameter.
    Suppose there is such a $p$: we will construct a family of balls disjoint from $\gamma$ with arbitrarily large closest point projections onto $\gamma$.
    We can assume without loss of generality that at $p$, one of the invariant two-sided curves is short.
    If not, we can create a new point $p^{\prime}$ where this is the case, and the orbits $\left\{ \gamma^i p \right\}$ and $\left\{ \gamma^i p^{\prime} \right\}$ have unbounded closed point projections onto each other, and if a family of balls has unbounded closest point projections on $\left\{ \gamma^i p^{\prime} \right\}$, it will also have unbounded closest point projections on $\left\{ \gamma^i p \right\}$.

    Since we have that some two-sided curve $\kappa$ is short at $p$, and $\gamma$ leaves $\kappa$ invariant, we have that the entire orbit $\left\{ \gamma^i p \right\}$ lies in the product region associated to $\kappa$.
    In this product region, we can verify via Minsky's product region theorem using a family of balls of radius $R$ centered at $q$, where $q$ is a point in the product region where the length of $\gamma$ is $\exp(-R)$ times smaller than the length of $\gamma$ at $p$.
    For each $R$, such a ball is disjoint from $\left\{ \gamma^i p \right\}$ will have projection diameter $2R - c^{\prime}$, where $c^{\prime}$ is some constant that only depends on $\vept$ (see \autoref{fig:non-contracting}).
    \begin{figure}[h]
      \centering
      \incfig[1]{non-contracting}
      \caption{Ball in product region with large projection onto axis.}
      \label{fig:non-contracting}
    \end{figure}
  \end{description}
  This proves the theorem in the two cases we specified.
\end{proof}

\begin{remark}
  The only case the above classification does not deal with is the case where $\gamma$ is a pseudo-Anosov on a subsurface that is the complement of only one-sided curves.
\end{remark}

%%% Local Variables:
%%% TeX-master: "main"
%%% End: