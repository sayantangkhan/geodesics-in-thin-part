\section{Geometry of $\teich(\no_g)$}
\label{sec:geom-of-teich}

In this section, we prove some standard results about the geometry of Teichmüller spaces of non-orientable surfaces that we use in Section \ref{sec:line-gap-compl}.
We do so by lifting the hyperbolic structures and markings on the non-orientable surfaces to their double covers, which give us points in the Teichmüller space and curve complex of the double cover.

The fact that these lifts are well-defined and respect the metric properties are encapsulated in the following two theorems.

\begin{theorem}[Isometric embedding of Teichmüller spaces (Theorem 2.1 of \cite{limitsetkhan})]
  \label{thm:i-embedding-teich-space}
  The map $i : \teich(\no_g) \to \teich(\os_{g-1})$ given by lifting the hyperbolic structure and marking from $\no_g$ to $\os_{g-1}$ is an isometric embedding.
  Furthermore, the image of $\teich(\no_g)$ in $\teich(\os_{g-1})$ is the subset of $\teich(\os_{g-1})$ is fixed by $\iota^{\ast}$, where $\iota^{\ast}$ is the map induced by the orientation reversing deck transformation $\iota$ on $\os_{g-1}$.
\end{theorem}

\begin{theorem}[Quasi isometric embedding of curve complexes (Lemma 6.3 from \cite{masur2013geometry})]
  \label{thm:qi-embedding-curve-complex}
  The map $\cC(\no_g) \to \cC(\os_{g-1})$ obtained by lifting curves in $\no_g$ to $\os_{g-1}$ is a quasi-isometric embedding.
\end{theorem}

We will use the above two theorems, along with Lemma \ref{lem:lifting-subsurfaces}, to reduce statements about the geometry of $\teich(\no_g)$ to statements about the geometry of $\teich(\os_{g-1})$.
However, we postpone the statement and the proof of Lemma \ref{lem:lifting-subsurfaces} until Section \ref{sec:dist-form-teichm}, since it's not required for Section \ref{sec:minskys-prod-regi}.

We set up some notation for this section.
\begin{itemize}
\item[-] $d(x,y)$ and $d(\wt{x}, \wt{y})$: Given points $x$ and $y$ in $\teich(\no_g)$, $d(x, y)$ is the distance in Teichmüller metric between them, and $d(\wt{x}, \wt{y})$ is the distance in $\teich(\os_{g-1})$ between their images, $\wt{x}$ and $\wt{y}$.
\item[-] $\pi_V(\mu_x)$ and $\pi_V(x)$: If $\mu_x$ is a marking/curve on a surface, the $\pi_V(\mu_x)$ denotes the subsurface projection to the subsurface $V$. If $x$ is a point in the Teichmüller space, the $\pi_V(x) = \pi_V(\mu_x)$, where $\mu_x$ is the Bers marking on $x$.
\item[-] $d_V(\mu_x, \mu_y)$ and $d_V(x, y)$: If $\mu_x$ and $\mu_y$ are markings/curves on a surface, and $V$ is a subsurface, then $d_V(\mu_x, \mu_y)$ refers to the curve complex distance between the subsurface projections of $\mu_x$ and $\mu_y$ in $\cC(V)$.
  When $x$ and $y$ are points in Teichmüller space, $d_V(x,y)$ refers to $d_V(\mu_x, \mu_y)$, where $\mu_x$ and $\mu_y$ are the Bers marking on $x$ and $y$.
\end{itemize}

% \todo[inline]{Set up the notation, i.e.\ the various metrics floating around.}

\subsection{Minsky's product region theorem}
\label{sec:minskys-prod-regi}

In this section, we prove a version of Minsky's product region theorem \cite[Theorem 6.1]{1077244446} for non-orientable surfaces.

% Let $\gamma = \left\{ \gamma_1, \ldots, \gamma_j, \ldots, \gamma_k \right\}$ be a multicurve on a non-orientable surface $\no_g$, where for $i \leq j$, $\gamma_i$ is a two-sided curve, and for $i > j$, $\gamma_i$ is a one-sided curve.
% Let $X_\gamma$ denote the product $\teich(\no_g \setminus \gamma) \times \mathbb{H}_1 \times \cdots \times \mathbb{H}_j \times (\mathbb{R}_{>0})_{j+1} \times \cdots \times (\mathbb{R}_{>0})_k$.
% For any pants decomposition that contains $\gamma$, we get a map $\Pi$ from $\teich(\no_g)$ to $X_\gamma$, which is called the \emph{product region projection map}.
% \begin{definition}[Product region projection map]
%   The product region projection map $\Pi: \teich(\no_g) \to X_\gamma$ is defined in the following manner.
%   \begin{itemize}
%   \item The $\teich(\no_g \setminus \gamma)$-coordinate is obtained by pinching all the curves in $\gamma$ to get a punctured hyperbolic surface.
%   \item The $\mathbb{H}_i$-coordinate is $\left( t, \frac{1}{\ell} \right)$, where $t$ is the \emph{twist} of the two-sided curve $\gamma_i$, and $\ell$ is the hyperbolic length.
%   \item The $(\mathbb{R}_{>0})_i$ coordinate is $\frac{1}{\ell}$, where $\ell$ is the hyperbolic length of the one-sided curve $\gamma_i$.
%   \end{itemize}
% \end{definition}

% We define a metric on the product $X_\gamma$ as the supremum of the metrics on each of the components, where the metric on $\teich(\no_g \setminus \gamma)$ is the Teichmüller metric, the metric on the $\mathbb{H}_i$ components is the hyperbolic metric, and the metric on the $(\mathbb{R}_{>0})_i$ is given by $d(x,y) = \left| \log\left( \frac{x}{y}  \right) \right| $, i.e.\ the restriction of the hyperbolic metric in $\mathbb{H}$ to a vertical line.

% We will be interested in looking at the restriction of $\Pi$ to the thin region of Teichmüller space, denoted $\teich_{\gamma \leq \vept}(\no_g)$, which is the region where all curves in $\gamma$ have hyperbolic length at most $\vept$.

We recall the following objects that were defined in Section \ref{sec:weak-conv-syst}.

\begin{enumerate}[(i)]
\item The multicurve $\gamma$ on $\no_g$.
\item The metric space $X_\gamma$, and the projection map $\Pi$.
\item The thin region $\teich_{\gamma \leq \vept}(\no_g)$.
\end{enumerate}

\begin{theorem}[Product region theorem for non-orientable surfaces]
  \label{thm:prno}
  For any $c >0$, there exists a small enough $\vept > 0$, such that the restriction of $\Pi$ to $\teich_{\gamma \leq \vept}(\no_g)$ is an isometry with additive error at most $c$, i.e.\ the following holds for any $x$ and $y$ in $\teich_{\gamma \leq \vept}(\no_g)$.
  \begin{align*}
    \left| d(x, y) - d_{X_{\gamma}}(\Pi(x), \Pi(y)) \right| \leq c
  \end{align*}
\end{theorem}

\begin{proof}
  We will prove this result by reducing the distance calculation in $\teich(\no_g)$ to a distance calculation in $\teich(\os_{g-1})$, where $\os_{g-1}$ is the orientation double cover, and invoking the classical product region theorem in that setting.

  We begin the proof by constructing some points in $\teich(\os_{g-1})$ and a multicurve on $\os_{g-1}$.
  Recall that $\teich(\no_g)$ isometrically embeds inside $\teich(\os_{g-1})$: let $\wt{x}$ and $\wt{y}$ denote the points in $\teich(\os_{g-1})$ that are the images of $x$ and $y$ under the embedding.
  Let $\wt{\gamma}$ denote the lift of the multicurve $\gamma$: if $\gamma_i$ is a two-sided curve, it will have two disjoint lifts in the cover, and if $\gamma_i$ is a one-sided curve, it will have single lift in the double cover.
  We have that the region $\teich_{\wt{\gamma} \leq \vept}(\os_{g-1}) \subset \teich(\os_{g-1})$ intersects the image of $\teich(\no_g)$ at the image of $\teich_{\gamma \leq \vept}(\no_g) \subset \teich(\no_g)$.
  Let $\iota$ denote the orientation reversing deck transformation on $\os_{g-1}$ which corresponds to the covering map.

  \begin{claim*}
    Let $\Pi_k$ denote the projection map from $\teich(\no_g)$ to the $k$\textsuperscript{th} component of $X_\gamma$, and $\wt{\Pi_k}$ denote the projection map from $\teich(\os_{g-1})$ to the lift of the $k$\textsuperscript{th} component of $\gamma$ to $\os_{g-1}$.
    This map is an isometric embedding.
    \begin{align*}
      d(\Pi_k(x), \Pi_k(y)) = d(\wt{\Pi_k}(\wt{x}), \wt{\Pi_k}(\wt{y}))
    \end{align*}
  \end{claim*}
  % \todo[inline]{Explain this better. Maybe have a claim here.}

  \begin{proof}[Proof of claim]
  We need to verify the claim on the three kinds of components of $\gamma$.
  \begin{enumerate}[(i)]
  \item $\no_g \setminus \gamma$: The lift of $\no_g \setminus \gamma$ to $\os_{g-1}$ will have two components if $\no_g \setminus \gamma$ is orientable, which we call $S_1$ and $S_2$. Both $S_1$ and $S_2$ are homeomorphic to $\no_g \setminus \gamma$.
    If $\no_g \setminus \gamma$ is non-orientable, then its lift in $\os_{g-1}$ is the orientation double cover.

    In the first case, $\teich(\no_g \setminus \gamma)$ maps to the diagonal subspace in $\teich(S_1) \times \teich(S_2)$, and the metric on $\teich(S_1) \times \teich(S_2)$ is the $\sup$ metric.
    The space $\teich(\no_g)$ maps to the diagonal subspace because its image must be invariant under the map $\iota$, which isometrically swaps $S_1$ and $S_2$.
    This map is an isometric embedding, and thus for any points $x$ and $y$ in $\teich(\no_g \setminus \gamma)$, the distance between their images in $\teich(S_1) \times \teich(S_2)$ is the same as the distance in $\teich(\no_g \setminus \gamma)$.

    In the second case, we have that $\teich(\no_g \setminus \gamma)$ also isometrically embeds inside the Teichmüller space of its double cover, by Theorem \ref{thm:i-embedding-teich-space}, so the claim follows.
  \item $\gamma_i$ (for $\gamma_i$ two-sided): The lift of $\gamma_i$ in this case are two disjoint curves on $\os_{g-1}$, which are swapped by the deck transformation $\iota$.
    This means the $\mathbb{H}$-coordinate given by length and twist of $\gamma_i$ maps to the diagonal in $\mathbb{H} \times \mathbb{H}$, which correspond the length and twist around the two lifts.
    Since $\mathbb{H}$ mapped to the diagonal in $\mathbb{H} \times \mathbb{H}$ is an isometric embedding with $\sup$ metric, the claim follows in this case.
  \item $\gamma_i$ (for $\gamma_i$ one-sided): The lift of $\gamma_i$ in this case is a single curve $\wt{\gamma_i}$ on $\os_{g-1}$ which is left invariant by the deck transformation $\iota$.
    We will show that the twist coordinate around $\wt{\gamma_i}$ cannot be changed without leaving the image of $\teich(\no_g)$ in $\teich(\os_{g-1})$, i.e.\ any $\wt{x}$ and $\wt{y}$ have the same twist coordinate around $\wt{\gamma_i}$.
    Once we have established that, the claim will follow, since only the length coordinate of $\gamma_i$ can be changed, which corresponds to $\mathbb{R}_{>0}$.

    Suppose now that $x$ is a point in $\teich(\no_{g})$ and $\wt{x}$ the corresponding point in $\teich(\os_{g-1})$.
    Consider a pants decomposition on $\no_g$ that contains $\gamma_i$ as one of the curves. There is a unique one-sided curve $\kappa$ that intersects $\gamma_i$ and does not intersect any of the other pants curves.
    Let $\wt{\kappa}$ be the lift of $\kappa$ to $\os_{g-1}$: we will use this curve to measure twisting around $\wt{\gamma_i}$.
    Let $x^{\prime}$ be another point in $\teich(\os_{g-1})$ obtained by taking $\wt{x}$, and twisting by some amount around $\wt{\gamma_i}$, without changing the length of $\wt{\gamma_i}$.
    On $x^{\prime}$, the length of $\wt{\kappa}$ will be different from the length on $\wt{x}$.
    However, this means that $x^{\prime}$ is not contained in the image of $\teich(\no_g)$, since if it were, the length of $\wt{\kappa}$ would have to be the same, since that's the lift of the curve $\kappa$, whose length only depends on the length of $\gamma_i$.
  \end{enumerate}
  \end{proof}

  The following equality follows from the claim.
  \begin{align*}
    d_{X_{\wt{\gamma}}}(\Pi(\wt{x}), \Pi(\wt{y})) = d_{X_{{\gamma}}}(\Pi({x}), \Pi({y}))
  \end{align*}
  We also have that $\teich(\no_g)$ isometrically embeds into $\teich(\os_{g-1})$.
  \begin{align*}
    d(\wt{x}, \wt{y}) = d(x, y)
  \end{align*}
  And finally, have that the region $\teich_{\wt{\gamma} \leq \vept}(\os_{g-1}) \subset \teich(\os_{g-1})$ intersects the image of $\teich(\no_g)$ at the image of $\teich_{\gamma \leq \vept}(\no_g) \subset \teich(\no_g)$.
  Combining these three facts, and applying Minsky's product region theorem for orientable surfaces, the result follows.
\end{proof}

% {\color{red} Move the theorem in Section 3 to this section}

\subsection{Uniform bounds for the volume of a ball}
\label{sec:unif-bounds-volume}

In this section, , we show that the for balls of fixed radius in $\core(\teich(\no_g))$, the $\nu_N$ volume of the ball is bounded above and below by constants that are independent of the center of the ball.

Let $\mathcal{P}$ be a pants decomposition for $\no_g$: recall the formula for $\nu_N$.

\begin{align*}
  \nu_N = \left( \bigwedge_{\text{$\gamma_i$ one-sided}} \coth(\ell(\gamma_i)) d\ell(\gamma_i) \right) \wedge \left( \bigwedge_{\text{$\gamma_i$ two-sided}} d\tau(\gamma_i) \wedge d\ell(\gamma_i) \right)
\end{align*}
Here $\ell(\gamma_i)$ denotes the length of the curve $\gamma_i$, and $\tau(\gamma_i)$ denotes the twist, when $\gamma_i$ is two-sided.

\begin{proposition}
  \label{prop:uniform-volume-bound}
  For any $\tau > 0$, and $\vept > 0$ small enough, there exist constants $c_1$ and $c_2$ (depending only on $\tau$ and $\vept$) such the $\nu_N$ volume of a ball $B_{\tau}^{\vept}(x)$ of radius $\tau$ centered at $x \in \systole(\no_g)$ are bounded below and above by $c_1$ and $c_2$.
  \begin{align*}
    c_1 \leq \nu_N(B_{\tau}^{\vept}(x)) \leq c_2
  \end{align*}
\end{proposition}
\begin{proof}
  Note that since the points we are considering lie in $\systole(\no_g)$, we have the following upper bound and lower bound for $\coth(\ell(\gamma_i))$, where $\gamma_i$ is a one sided curve.
  \begin{align}
    \label{eq:coth-bounds}
    1 \leq \coth(\ell(\gamma_i)) \leq \coth(\vept)
  \end{align}
  In particular, the $\nu_N$ volume of a ball can be bounded above and below by $\coth(\vept)\nu_N^{\prime}$ and $\nu_N^{\prime}$, where $\nu_N^{\prime} = \left( \bigwedge_{\text{$\gamma_i$ one-sided}} d\ell(\gamma_i) \right) \wedge \left( \bigwedge_{\text{$\gamma_i$ two-sided}} d\tau(\gamma_i) \wedge d\ell(\gamma_i) \right)$.

  We now split up $\systole(\no_g)$ into two regions: $\thick(\no_g)$, and the complementary region.
  Since $\mcg(\no_g)$ acts cocompactly on $\thick(\no_g)$, and $\nu_N(B_{\tau}^{\vept}(x))$ is continuous in $x$, the desired bounds hold in this region.
  It will therefore suffice to prove the bounds in the complementary region.

  Note that for any $x$ in the complementary region, there is some two-sided curve $\gamma$ that is short.
  By Theorem \ref{thm:prno}, the ball $B_{\tau}^{\vept}(x)$ is contained in a product of balls, one in $\mathbb{H}$, and one in $\systole(\no_g \setminus \gamma)$.
  We pick $\gamma$ to be part of a pants decomposition $\mathcal{P}$, and write $\nu_N$ as follows.
  \begin{align*}
    \nu_N = \left( d\tau(\gamma) \wedge d\ell(\gamma) \right) \wedge \nu_N^{\no_g \setminus \gamma}
  \end{align*}
  Here, $\nu_N^{\no_g \setminus \gamma}$ denotes the volume form on $\teich(\no_g \setminus \gamma)$.
  As a result, we have that the $\nu_N$ measure of a product of the two balls is the product of the correspond measures of those balls.

  The measure of any ball of a fixed radius in $\mathbb{H}$ is constant, since $\mathbb{H}$ is homogeneous.
  The $\nu_N^{\no_g \setminus \gamma}$ measure of a ball in $\systole(\no_g \setminus \gamma)$ is again bounded above and below by fixed constants, by inducting on a surface of lower complexity.

  Since we have uniform bounds for both the terms in the product, we get uniform bounds for the measure of a ball in $\systole(\no_g)$.
\end{proof}

\subsection{Teichmüller geodesics and geodesics in the curve complex}
\label{sec:dist-form-teichm}

In this section, we will deduce some standard results about Teichmüller geodesics and the corresponding curve complex geodesics for non-orientable surfaces by reducing to the orientable case.
The following lemma will be the main tool for the reduction to the orientable case.
\begin{lemma}
  \label{lem:lifting-subsurfaces}
  Let $[x, y]$ be a Teichmüller geodesics segment in $\teich(\no)$, where $\no$ is a non-orientable surface, and $[\wt{x}, \wt{y}]$ be its image in $\teich(\os)$, where $\os$ is the orientable double cover of $\no$.
  Let $V$ be a subsurface of $S$: then the following for $d_V(\wt{x}, \wt{y})$.
  \begin{enumerate}[(i)]
  \item If $V$ is the lift of a non-orientable subsurface $W$ in $\no$, then $d_V(\wt{x}, \wt{y}) \emul d_W(x, y)$.
  \item If $V$ is the lift of an orientable subsurface $W$ in $\no$, then $d_V(\wt{x}, \wt{y}) = d_{\iota(V)}(\wt{x}, \wt{y}) = d_W(x, y)$.
  \item If $V$ is not a lift of a subsurface in $\no$, then there exists a uniform constant $k_0$, independent of $x$, $y$, and $V$, such that $d_V(\wt{x}, \wt{y}) \leq k_0$.
  \end{enumerate}
\end{lemma}
\begin{proof}
  We deal with the proof in cases.
  \begin{enumerate}[(i)]
  \item If $V$ is the lift of an orientable surface, we have that the covering map restricted to $V$ is a homeomorphism, and the same holds for $\iota(V)$, so the result follows in this case as well.
  \item If $V$ is the lift of a non-orientable subsurface $W$, then by Theorem \ref{thm:qi-embedding-curve-complex}, we have that $\cC(W)$ quasi-isometrically embeds into $\cC(V)$, and the result follows.
  \item If $V$ is not a lift at all, that means $V$ and $\iota(V)$ are transverse subsurfaces.
    By the Behrstock inequality, there exists a $k_0$ such that the following holds.
    \begin{align}
      \label{ineq:min-of-equal}
      \min(d_V(\wt{x}, \wt{y}), d_{\iota(V)}(\wt{x}, \wt{y})) \leq k_0
    \end{align}
    Note that we also have that $\wt{x}$ and $\wt{y}$ are fixed by $\iota^{\ast}$, the induced map on $\teich(\os)$, which gives us the following.
    \begin{align*}
      d_{\iota(V)}(\wt{x}, \wt{y}) &= d_{V}(\iota^{\ast}(\wt{x}), \iota^{\ast}(\wt{y})) \\
                                   &= d_V(\wt{x}, \wt{y})
    \end{align*}
    This means that both the terms appearing in the $\min$ in \eqref{ineq:min-of-equal} are equal, which gives us the result.
  \end{enumerate}
\end{proof}

% \todo[inline]{Set up some context.}
We begin by proving the distance formula for points in Teichmüller space.
Let $x$ and $y$ be a pair of points in $\teich(\no_g)$, and let $\Gamma$ be the set of curves that are short on both $x$ and $y$, $\Gamma_x$ the set of curves that are only short on $x$, and $\Gamma_y$ the set of curves that are only short on $y$.
Let $\mu_x$ and $\mu_y$ be short markings on $x$ and $y$ respectively.
Let $\cC^+$ and $\cC^-$ denote the set of two-sided and one-sided curves on $\no_g$.
Finally, let $\left[ x \right]_k$ be the function which is $0$ for $x \leq k$, and identity for $x > k$.

\begin{theorem}[Distance formula]
  \label{thm:distance-formula}
  The distance between $x$ and $y$ in $\teich(\no_g)$ is given by the following formula.
  \begin{equation}
  \begin{aligned}
    d(x, y) &\emul \sum_{Y} \left[ d_Y(\mu_x, \mu_y) \right]_k + \sum_{\alpha \in \Gamma^c \cap \cC^+} \left[ \log(d_{\alpha}(\mu_x, \mu_y)) \right]_k \\
    &+ \max_{\alpha \in \Gamma \cap \cC^+} d_{\mathbb{H}_{\alpha}}(x, y) + \max_{\alpha \in \Gamma \cap \cC^-} d_{(\mathbb{R}_{>0})_{\alpha}}(x, y) \\
    &+ \max_{\alpha \in \Gamma_x} \log \frac{1}{\ell_x(\alpha)} + \max_{\alpha \in \Gamma_y} \log \frac{1}{\ell_y(\alpha)}
  \end{aligned}
  \label{eq:distance-formula-nonorient}
  \end{equation}
\end{theorem}
\begin{proof}
  Let $\wt{x}$ and $\wt{y}$ be the images of $x$ and $y$ in $\teich(\os_{g-1})$ under the isometric embedding map.
  Since $d(x,y) = d(\wt{x}, \wt{y})$, it will suffice to estimate $d(\wt{x}, \wt{y})$ using distances in the curve complexes.
  Let $\wt{\mu_x}$ and $\wt{\mu_y}$ be the lifts of $\mu_x$ and $\mu_y$.
  Both $\wt{\mu_x}$ and $\wt{\mu_y}$ are short markings on $\wt{x}$ and $\wt{y}$ respectively.
  We have by Rafi's distance formula \cite[Theorem 6.1]{rafi2007combinatorial}, the following estimate on $d(\wt{x}, \wt{y})$.
  \begin{equation}
  \begin{aligned}
    d(\wt{x}, \wt{y}) &\emul \sum_{Y} \left[ d_Y(\wt{\mu_x}, \wt{\mu_y}) \right]_k + \sum_{\alpha \in \wt{\Gamma}^c} \left[ \log(d_{\alpha}(\wt{\mu_x}, \wt{\mu_y})) \right]_k \\
    &+ \max_{\alpha \in \wt{\Gamma}} d_{\mathbb{H}_{\alpha}}(\wt{x}, \wt{y})\\
    &+ \max_{\alpha \in \wt{\Gamma_{\wt{x}}}} \log \frac{1}{\ell_{\wt{x}}(\alpha)} + \max_{\alpha \in \Gamma_{\wt{y}}} \log \frac{1}{\ell_{\wt{y}}(\alpha)}
  \end{aligned}
  \label{eq:distance-formula-orient}
  \end{equation}
  Here, $\wt{\Gamma}$, $\wt{\Gamma_{\wt{x}}}$, and $\wt{\Gamma_{\wt{y}}}$ are curves on $\wt{x}$ and $\wt{y}$ that are simultaneously short, short on $\wt{x}$ and not on $\wt{y}$, and short on $\wt{y}$ and not on $\wt{x}$ respectively.

 It will suffice to show that for a large enough choice of $k$, the right hand side of \eqref{eq:distance-formula-nonorient} is equal to the right hand side of \eqref{eq:distance-formula-orient}, up to an additive and multiplicative constant.
 We consider the first term in the right hand side of \eqref{eq:distance-formula-orient}, namely the sum over the non-annular subsurfaces $Y$.
 There are three possibilities for $Y$ in $\os_{g-1}$, which we deal with using Lemma \ref{lem:lifting-subsurfaces}.
 \begin{enumerate}[(i)]
 \item $Y$ is one component of a lift of an orientable subsurface $Z$ of $\no_g$: In this case we have $d_Y(\wt{\mu_x}, \wt{\mu_y}) = d_Z(\mu_x, \mu_y)$ (and the same equality with $Y$ replaced with $\iota(Y)$).
   Thus, for every term associated to an orientable non-annular subsurface $Z$ in \eqref{eq:distance-formula-nonorient}, we get two corresponding equal terms in \eqref{eq:distance-formula-orient}.
 \item $Y$ is the lift of a non-orientable subsurface $Z$ of $\no_g$: In this case, we have $d_Y(\wt{\mu_x}, \wt{\mu_y}) \emul d_Z(\mu_x, \mu_y)$.
 \item $Y$ is not a lift of a subsurface of $\no_g$: In this case, we have the following for some $k_0$.
   \begin{align*}
     d_Y(\wt{\mu_x}, \wt{\mu_y}) \leq k_0
   \end{align*}
   If we pick a threshold $k > k_0$, the subsurfaces $Y$ that do not arise from lifts will not contribute to the right hand side of \eqref{eq:distance-formula-orient}.
 \end{enumerate}
 % The analysis of the three possible types of non-annular subsurfaces $Y$ shows that the first terms on the right hand side of \eqref{eq:distance-formula-nonorient} and \eqref{eq:distance-formula-orient} are equal up to a multiplicative and additive constant, for $k > k_0$.

 We now do the same case analysis for annular subsurfaces: consider a curve $\alpha$ on $\os_{g-1}$ that is contained in $\Gamma^c$, i.e.\ it is not simultaneously short on $\wt{x}$ and $\wt{y}$. There are three possibilities for $\alpha$.
 \begin{enumerate}[(i)]
 \item $\alpha$ is one component of a lift of a two-sided curve $\gamma$ on $\no_g$: In this case, $\alpha$ and $\iota(\alpha)$ are disjoint, and the restriction of the covering map to these curves is a homeomorphism.
   We have $d_\alpha(\wt{\mu_x}, \wt{\mu_y}) = d_\gamma(\mu_x, \mu_y)$: consequently, for every term in $\Gamma^c \cap \cC^+$ in \eqref{eq:distance-formula-nonorient}, we have two equal terms in \eqref{eq:distance-formula-orient}.
 \item $\alpha$ is the lift of a one-sided curve on $\no_g$: In this case $\alpha = \iota(\alpha)$, but the transformation $\iota$ reverses orientation on the surface $\os_{g-1}$.
   That means $\wt(\mu_x)$ and $\wt{\mu_y}$ cannot have a relative twist between them along $\alpha$, because if they did, $\iota(\wt(\mu_x))$ and $\iota(\wt(\mu_y))$ would have the opposite twist.
   On the other hand $\wt{\mu_i} = \iota(\wt{\mu_i})$ for $i=x$ and $i=y$, which means the relative twist must be $0$.
   This proves that the $\alpha$ which are lifts of one-sided curves do not contribute to the second term of \eqref{eq:distance-formula-orient}.
 \item $\alpha$ is not a lift of a curve on $\no_g$: In this case $\alpha$ and $\iota(\alpha)$ intersect each other, and are not equal, which means they are transverse.
   We deal with this the same way we dealt with transverse non-annular subsurfaces, i.e.\ via the Behrstock inequality.
 \end{enumerate}
 This case analysis proves that the second terms on the right hand side of \eqref{eq:distance-formula-nonorient} and \eqref{eq:distance-formula-orient} are equal, up to an additive and multiplicative constant.

We now deal with the last three terms of \eqref{eq:distance-formula-orient}.
These terms deal with short curves on $x$ or $y$: we claim that the short curves must be lifts of either one-sided or two-sided curves in $\no_g$.
Suppose a curve $\alpha$ is short and not a lift. Then $\alpha$ has positive intersection number with $\iota(\alpha)$, but since $\iota$ is an isometry, $\iota(\alpha)$ must also be short.
For a sufficiently small threshold for what we call short, we can't have a short curve intersecting another short curve, which proves the claim that all the short curves arise as lifts.

Since the curves in $\wt{\Gamma}$ are all lifts, the third term of \eqref{eq:distance-formula-orient} can be split up into two terms: the lifts of the two-sided and one-sided curves.
For the two-sided curves, the distance calculation involves both the length and twist coordinate, and for the one-sided curves, only the length coordinate is involved.
This follows from Theorem \ref{thm:prno}.

Finally, the last two terms in \eqref{eq:distance-formula-orient} are the same as the last two terms of \eqref{eq:distance-formula-nonorient}, up to an additive error of $(6g) \cdot \log(2)$, since the lift of a short curve can double its length, and there are no more than $6g$ short curves.

We have shown that the right hand sides of \eqref{eq:distance-formula-nonorient} and \eqref{eq:distance-formula-orient} are equal, up to a multiplicative and additive constant, which proves the result.
\end{proof}

We now verify that Teichmüller geodesics can be broken up into \emph{active intervals associated to subsurfaces}, which are subintervals of the geodesic associated to each subsurface $V$, along which the projection to $V$ is large, and outside of which, the projection is bounded.
% To build up the machinery of complexity length, we will also need a description of the Teichmüller geodesic segments.
% Let $[x, y]$ be a Teichmüller geodesic segment between $x$ and $y$ in $\teich(\no_g)$.
The following lemma  of \textcite[Lemma 3.26]{dowdall2023lattice} (which itself is a generalization of \textcite[Proposition 3.7]{rafi2007combinatorial}) describes the subsegments of $[x,y]$ along which the geodesic makes progress in the curve complex of a subsurface.

\begin{proposition}
  \label{thm:active-intervals}
  For each sufficiently small $\vept > 0$, there exists $0 < {\vept}^{\prime} < \vept$ and $M_{\vept} \geq 0$ such that for any subsurface $V \sqsubset S$, there's a (possibly empty) connected interval $\mathcal{I}_V^{\vept} \subset [x,y]$ such that the following five conditions hold.
  \begin{enumerate}[(i)]
  \item If $d_V(x, y) \geq M_{\vept}$, then $\mathcal{I}_V^{{\vept}}$ is a non-empty subinterval of $[x,y]$.
  \item $\ell_\alpha(z) < {\vept}$ for all $z \in \mathcal{I}_V^{{\vept}}$ and $\alpha \in \partial V$.
  \item For all $z \in [x,y] \setminus \mathcal{I}_V^{{\vept}}$, some component $\alpha$ of $\partial V$ has $\ell_\alpha(z) > {\vept}^{\prime}$.
  \item $d_V(w, z) \leq M_{\vept}$ for every subinterval $[w,z] \subset [x,y]$ if $[w,z] \cap \mathcal{I}_V^{\vept} = \varnothing$.
  \item For a pair of traverse subsurfaces $U$ and $V$, $\mathcal{I}_U^{\vept} \cap \mathcal{I}_V^{\vept} = \varnothing$.
  \end{enumerate}
\end{proposition}

\begin{proof}[Proof of Theorem \ref{thm:active-intervals} for non-orientable surfaces]
  Let $\no$ be the non-orientable surface, and $\os$ its double cover.
  We consider the image $[\wt{x}, \wt{y}]$ of the geodesic $[x,y]$ in $\teich(\os)$.
  We know that the result holds for $[\wt{x}, \wt{y}]$, although with ${\vept}^{\prime}$ replaced with $\frac{{\vept}^{\prime}}{2}$, since lifting can double the lengths of some curves.

  The main fact we need to verify is that the only subsurfaces $V$ that have non-empty $\mathcal{I}_V^{\vept}$ come from lifts.
  If $V$ is a subsurface of $\os$ that is not a lift, we use case (iii) of Lemma \ref{lem:lifting-subsurfaces} to conclude that $d_V(x, y) \leq k_0$ for some fixed constant $k_0$.
  Picking $M_{\vept} > k_0$ guarantees that the only subsurfaces for which $\mathcal{I}_V^{\vept}$ is non-empty arise from lifts, which proves the result for non-orientable surfaces.
\end{proof}

Finally, we show that the consistency and the realization results (\textcite{behrstock2012geometry}) hold for Teichmüller spaces of non-orientable surfaces as well.
We begin by recalling the definition of consistency.

\begin{definition}[Consistency]
  For a connected surface $S$, and a parameter $\theta \geq 1$, we say a tuple $(z_V) \in \prod_{V \sqsubset S} \cC(V)$ is $\theta$-consistent if the following two conditions holds for all pairs of subsurfaces $U$ and $V$.
  \begin{enumerate}[(i)]
  \item If $U \pitchfork V$, then
    \begin{align*}
      \min(d_U(z_U, \partial V), d_V(z_V, \partial U)) \leq \theta
    \end{align*}
  \item If $U \sqsubset V$, then
    \begin{align*}
      \min(d_U(z_U, \pi_U(z_V)), d_V(z_V, \partial U)) \leq \theta
    \end{align*}
  \end{enumerate}
\end{definition}
% \todo[inline]{Make sure I use the transversality notation consistently.}

The following theorem (\textcite[Theorem 4.3]{behrstock2012geometry}) states that the projection from Teichmüller space to the curve complexes of all the subsurfaces is coarsely surjective onto the set of consistent tuples.

\begin{theorem}[Consistency and realization]
  \label{thm:consistency-realization}
  There is a constant $K \geq 1$, and function $\mathfrak{C}: \mathbb{R}_+ \to \mathbb{R}_+$ such that the following holds for any surface $S$.
  \begin{itemize}
  \item (Consistency) For every $x \in \teich(S)$, the projection tuple $(\pi_V(x))_{V \sqsubset S}$ is $K$-consistent.
  \item (Realization) For every $\theta$-consistent tuple $(z_V)_{V \sqsubset S}$, there exists a point $z \in \teich(S)$ such that $d_V(\pi_V(z), z_V) \leq \mathfrak{C}(\theta)$ for all $V$.
  \end{itemize}
\end{theorem}
\begin{proof}[Proof sketch of Theorem \ref{thm:consistency-realization} for non-orientable surfaces]
    We first show that the projection map is consistent, and then show consistent tuples lie coarsely in the image of the projection map.
  \begin{itemize}
  \item (Consistency) We map $x$ to $\wt{x}$ in the Teichmüller space of the double cover $\wt{S}$.
    By applying the theorem for orientable surfaces, we have the $(\pi_{W}(\wt{x}))_{W \sqsubset \wt{S}}$, and we restrict to the subsurfaces in the tuple which arise as lifts.
    These points lie in the image of the quasi-isometric embedding map from Theorem \ref{thm:qi-embedding-curve-complex}, which means consistency also holds for the tuples in $S$.
  \item (Realization) Given a $\theta$-consistent tuple $(z_V)_{V \sqsubset S}$, we construct a $\theta^{\prime}$-consistent tuple in the double cover $\wt{S}$.
    For subsurfaces of $\wt{S}$ that arise as lifts, we use the map from Theorem \ref{thm:qi-embedding-curve-complex}.
    For the subsurfaces $W$ that are not lifts, we set $z_W = \pi_W(\partial(\iota(W)))$.
    The fact that this is a $\theta^{\prime}$-consistent tuple follows from the Behrstock inequality (for some $\theta^{\prime} > \theta$)\footnote{A longer but a more thorough way of seeing this would be to verify that the Teichmüller space of a non-orientable surface satisfies the $9$ axioms for hierarchical hyperbolicity that are enumerated in \textcite{behrstock2019hierarchically}.}.
    We now use this point to construct $y \in \teich(\wt{S})$, and deduce that $y$ is coarsely fixed by $\iota^{\ast}$.
    This means there is some $x \in \teich(S)$ whose image is coarsely $y$, and therefore the projection maps are coarsely $(z_V)$.
  \end{itemize}
\end{proof}

%%% Local Variables:
%%% TeX-master: "main"
%%% End: