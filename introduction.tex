\section{Introduction}
\label{sec:introduction}

{\color{red} Outline of introduction.}
\begin{enumerate}[(i)]
\item List similarities with IVGF surfaces, and point out where analogy breaks down.
\item State theorem about systole being weak convex.
\item Weak convexity lets us focus on systole, rather than entire space.
\item Can prove exponential rareness of thin geodesics in systole, using random walk methods.
\item Using this, we get stat-convex, and consequently, rest of PS theory thanks to others.
\item Hard part of random walk argument is showing hLP = hNP. We spend sections 5 and 6 doing this.
\item Sketch out idea of proof of hLP = hNP.
\item Mention what's in the appendix.
\end{enumerate}

Mapping class groups $\mcg(\no_g)$ of non-orientable surfaces $\no_g$ have many similarities to infinite co-volume geometrically finite Fuchsian groups, in a manner similar how mapping class groups of orientable surfaces behave like lattices in $\mathrm{SL}_2(\mathbb{R})$.

\begin{enumerate}[(i)]
\item The mapping class groups $\mcg(\no_g)$ are finitely presented.
\item The action of $\mcg(\no_g)$ on the Teichmüller space $\teich(\no_g)$, has infinite $\nu_N$-covolume, where $\nu_N$ is the generalization of the Weil-Petersson volume form on Teichmüller spaces of non-orientable surfaces (\cite[Theorem 17.1]{gendulphe2017whats} and \cite{norbury2008lengths}).
\item The limit set of $\mcg(\no_g)$ in the Thurston boundary of $\teich(\no_g)$ is $\pml^+(\no_g)$, i.e. the projective measured laminations that have no two-sided components, which is a subset of the boundary with zero Lebesgue measure (\cite{erlandsson2023mapping} and \cite{limitsetkhan}).
\item The Teichmüller geodesic flow is not ergodic with respect to any Borel measure on the unit cotangent bundle with full support (\cite[Proposition 17.5]{gendulphe2017whats})
\item There exists an $\mcg(\no_g)$-equivariant deformation retract of $\teich(\no_g)$ to $\systole(\no_g)$, the subset where no one-sided curve is shorter than $\vept > 0$. The action of $\mcg(\no_g)$ on $\systole(\no_g)$ is finite $\nu_N$-covolume \cite[Proposition 19.1]{gendulphe2017whats}.
\end{enumerate}

With these similarities, one might expect that $\systole(\no_g)$ serves as a convex core of $\teich(\no_g)$, and the geodesic flow restricted to tangent directions whose forward and backward end points lie in the limit set will be ergodic, with respect to a finite geodesic flow invariant measure supported only on these tangent directions.

A prior result of the author shows that this is not the case, i.e. $\systole(\no_g)$ is not even quasi-convex, i.e. geodesic segments whose endpoints lie in $\systole(\no_g)$ can travel arbitrarily far from $\systole(\no_g)$.

\begin{theorem}[Theorem 5.2 of \cite{limitsetkhan}]
  For all $\vept > 0$, and all $D > 0$, there exists a Teichmüller geodesic segment whose endpoints lie in $\systole(\no_g)$, such that some point in the interior of the geodesic segment is more than distance $D$ from $\systole(\no_g)$.
\end{theorem}

\subsection*{Organization of the paper}


\tableofcontents

% \todo[inline]{Find and rewrite all phrases that begin with ``it's easy to see that''.}

%%% Local Variables:
%%% TeX-master: "main"
%%% End: